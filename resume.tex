\documentclass{article}
\usepackage[a4paper,bottom = 0.3in,left = 0.4in,right = 0.4in,top = 0.75in]{geometry}
\usepackage{graphicx}
\usepackage{amsmath, bm}
\usepackage{array}
\usepackage{enumitem}
\usepackage{wrapfig}
\usepackage{titlesec}
\usepackage[hidelinks]{hyperref}

\newcommand{\xfilll}[2][1ex]{
\dimen0=#2\advance\dimen0 by #1
\leaders\hrule height \dimen0 depth -#1\hfill}
\titleformat{\section}{\large\scshape\raggedright}{}{0em}{}
\renewcommand\labelitemi{\raisebox{0.4ex}{\tiny$\bullet$}}
\renewcommand{\labelitemii}{$\cdot$}
\pagenumbering{gobble}
\begin{document}
\vspace*{120 pt}
% \vspace{-10pt}
% \hspace*{-14pt}
%Pursuing {\bf Major} in Computer Science and Engineering \\
\vspace*{-20 pt}

%%%%%%%%%%%%%%%%%%%%%%%%%%%%%%%%%%%%%%%%%%%%%%%%%%%%%%%%%%SCHOLASTIC ACHIEVEMENTS %%%%%%%%%%%%%%%%%%%%%%%%%%%%%%%%%
\vspace{-2pt}
\section*{{\LARGE Scholastic Achievements}\xfilll[0pt]{0.5pt}}
\vspace{-8pt}
\begin{itemize}[itemsep = -1 mm, leftmargin=*]
% {\bf 8\textsuperscript{th} in the Department} out of 121 students and
%\item Ranked {\bf 10\textsuperscript{th} in the Institute} out of 905 students\hfill{\sl \small (2016)}
\item Secured {\bf All India Rank 4}, State Rank 2 in the Grade 10 {\bf ICSE} Examination out of {\bf 170,000} students \hfill{\sl \small (2016)}
\item Achieved {\bf All India Rank 5} in the {\bf CBSE} Board Examinations (Grade 12) out of {\bf 1.2 million}\\ candidates ({\it {\bf Overall} East Zone Topper} and {\it All India Rank 3} in the Science Stream)\hfill{\sl \small (2018)}
\item \underline{Only student} to receive the {\bf Advanced Performance}(AP) grade in {\bf Computer Programming} \& {\bf Utilization} and amongst the {\bf top 3} in the {\bf Biology} course\textit{(Molecular, Physical and Biomedical modules)} \textsl{out of \textbf{500+} students}\hfill{\sl \small(2018-19)}
\item Awarded {\bf AP grade} in {\bf Advanced Calculus} (\textsl{top} {\bf 12} \textsl{out of} {\bf 1000+} students) for \textit {exemplary} performance\hfill{\sl \small(2018)}
% \item Awarded {\bf AP grade} in {\bf Biology} course consisting of Molecular, Physical and Biomedical modules (given to top 3 students out of 502)\hfill{\sl \small(2018)}
\item Attained a Semester Performance Index (\textbf{SPI}) of \textbf {perfect 10} in the First Semester \hfill{\sl \small(2018)}
\item Among the {\bf top 12} students to be granted {\bf Change of Branch}/Major to Computer Science\hfill{\sl \small(2019)}
\item Offered {\bf Computer Science} at the National University of Singapore ({\bf NUS}) with \textbf{100 \%} scholarship\hfill{\sl \small(2018)}
\end{itemize}

\vspace{-20pt}
\section*{{\LARGE Scholarships and Recognition}\xfilll[0pt]{0.5pt}}
\vspace{-8pt}
\begin{itemize}[itemsep = -1 mm, leftmargin=*]
\item  Bagged the {\bf Institute Academic Award}, given to the \textbf{Top 25} out of a batch of 1000+ students for {\it exceptional}\\ academic performance in the first year of Undergraduate Study at IIT Bombay
 \hfill{\sl \small (2018-19)}
\item  Bestowed with the {\bf KVPY} (Kishore Vaigyanik Protsahan Yojna) Fellowship, given to the talented young \\minds in the field of Science and Technology, by Department of Science and Technology, Govt. of India \hfill{\sl \small (2018)}
\item Felicitated by \textbf{The Governor of West Bengal}, with the \textit{Mamraj Agarwal Rashtriya Puraskar} for exemplary performance in the ICSE and by \textit{Mr. S.K.Birla}, industrialist \& trustee of Birla High School with a \textbf{Gold} medal \hfill{\sl \small (2016, 2018)}
\item Received a \textbf{Letter of Appreciation} from Ms. Mamata Banerjee, Chief Minister of  West Bengal for \textit{exemplary} performance in the CBSE Examinations along with the \textit{\underline{Swami Vivekananda Scholarship}} for \textit{Undergraduate Study}\hfill{\sl \small (2018)}
\item Granted the \textit{Ramawatar Gupt Pratibha Puraskar} and a cash award by \textbf{Sanmarg Foundation} for securing \textbf{99}\% in Hindi in the ICSE, the \textbf{Times of India EduShine} for stupendous performance in the Grade 12 Board \hfill{\sl \small (2016, 2018)}
%\item Granted the {\it Ramawatar Gupt Pratibha Puraskar} and a cash award by {\bf Sanmarg Foundation} for securing {\bf99} \% in Hindi in the ICSE Board Examinations \hfill{\sl \small (2016)} 
%\item Felicitated with the {\bf Times of India EduShine} for stupendous performance in the Grade 12 Board \hfill{\sl \small (2018)} 
%\item  Recipient of the {\bf NTSE} (National Talent Search Examination) Scholarship by N.C.E.R.T. New Delhi \hfill{\sl \small (2016)}
% \item  Awarded the {\bf RS Pandey Proficiency Medal} by La Martiniere For Boys for securing the \textbf{highest marks} in the country in Hindi in the ICSE Examination \hfill{\sl \small (2016)}
\item  Recipient of the {\bf Udbhav Poddar Memorial} Prize and the Dr. {\bf RS Pandey Proficiency} {\it Silver Medal} for \\securing the \textbf{highest marks} in the country in Mathematics and Hindi respectively, in the ICSE Examinations\hfill{\sl \small (2016)}
%\item Felicitated with a {\bf Gold} medal, by {\it Mr. S.K.Birla}, industrialist \& trustee of Birla High School \hfill{\sl \small (2018)} 
\end{itemize}
%%%%%%%%%%%%%%%%%%%%%%%%%%%%%%%%%%%%%%%%%%%%%%%%%%%%%%%%%%  INTERESTS   %%%%%%%%%%%%%%%%%%%%%%%%%%%%%%%%%%%%%%%%%%

%\section*{{\LARGE Interests}\xfilll[0pt]{0.5pt}}
%\vspace{-4pt}
%Data Structures and Algorithms, Data Analysis and Interpretation, Graph Theory, Competitive Coding, Combinatorics and Probability Theory, Discrete Structures

%%%%%%%%%%%%%%%%%%%%%%%%%%%%%%%%%%%%%%%%%%%%%%%%%%%%%%%%%%  PROJECTS  %%%%%%%%%%%%%%%%%%%%%%%%%%%%%%%%%%%%%%%%%%%
\vspace{-20pt}
\section*{\LARGE Internship and Research Experience\xfilll[0pt]{0.5pt}}
\vspace{-8pt}
\textbf{Research Intern | Cryptography} \hfill{\sl \small (April - May '20)}\\
{\it Guide(s): Prof. Steve Kremer and Jannik Dreier}\hfill{\sl \small \textsc{INRIA, Nancy, France}}\\
\vspace{-18pt}
\begin{itemize}[itemsep = -1 mm, leftmargin=*]
    \item \underline{Formal {\bf Verification} of \textsl{security protocols}}: Studied operational semantics, equivalence properties (in \textit{applied pi calculus} and \textbf{Tamarin} prover) and \textbf{SAPIC} (tool translating high level protocols to multiset rewrite rules, {\it analyzable} by Tamarin)
    \item Introduced biprocesses (\textit{with bisemantics and their translation}), static equivalence and diff equivalence (\& \textit{diff} operator) in SAPIC, ensuring the soundness of the translation after the addition 
\end{itemize}
\vspace{-4pt}
\textbf{Quantitative Research Analyst} \hfill{\sl \small (Dec '19 - Jan '20)}\\
{\it Guide(s): Prof. Prasanna Tantri, Prof. Nitin Kumar and Ravi Ranjan  }\hfill{\sl \small \textsc{Indian School of Business, Hyderabad}}\\
\vspace{-18pt}
\begin{itemize}[itemsep = -1 mm, leftmargin=*]
    \item \textbf{Deep Learning: }\underline{\it Applying NLP techniques to \textbf{Time Series Analysis} for Stock Futures :}
    \vspace{-7pt}
    \begin{itemize}[itemsep = -0.75 mm, leftmargin=*]
      %\begin{small}
      \item Designed and implemented an intuitive approach to storing the history of a stock in the form of a vector using a Ticker \textbf{Embedding Model}, similar to a Word Embedding model. Incorporated technical indicators such as Momentum, Trailing Volatility, Asset Class and average return per asset class along with the embeddings
     \item Designed, trained and tested an \textbf{LSTM} classifier (using \textbf{PyTorch}) trained on a time series of multiple stock tickers to predict returns and study non linearity \& inter asset class correlation. 
     \item Expanded the LSTM to incorporate \textbf{attention}, and \textit{retrain} over latest data \textit{while testing}
     \item Optimized the hyperparameters using libraries: Ray for \textbf{Grid Search} and Hyperopt for \textbf{Bayesian} optimization
     %\end{small}
    \end{itemize}
    \vspace{-4pt}
    \item \textbf{Trading Algorithms:} Implemented the \textbf{Pairs}, \textbf{Betting against} $\bm{\beta}$ and \textbf{Momentum} trading algorithms in \textbf{Python}
    \vspace{-7pt}
    \begin{itemize}[itemsep = -0.75 mm, leftmargin=*]
    %\begin{small}
      \item Calculated $\beta$ by regression on the \textbf{CAPM} equation with a 6 month rolling window: Experimented with daily, weekly and monthly \textit{rebalancing} of the portfolio, and anaylzed the output on equally weighted and value weighted portfolios
     \item Modified the Pairs strategy on a 1 year rolling window with 12\% CAGR and 0.71 overall Sharpe and researched the intricacies involved in the strategies:  Pitroski's F-Score, Mohanram's G-Score, Accruals, PEAD and Momentum crashes
     %\end{small}
    \end{itemize}
\end{itemize}
\vspace{-5pt}
\textbf{Data Analytics Intern} \hfill{\sl \small (June - July '19)}\\
{\it Guide: Mr. Amit Ambekar (Vice President, Marketing)}\hfill{\sl \small \textsc{Spencer's Retail Ltd.- RPSG group}}\\
\vspace{-18pt}
\begin{itemize}[itemsep = -1 mm, leftmargin=*]
    \item {\it \underline {Analysis of underperforming stores} given all KPIs and SKU (Stock Keeping Units) level data}: {\bf Statistical Analysis} of transactional \& brick level data to attribute reasons for de-growth in the {\it MGF Gurgaon} and {\it Vizag} hyper stores
     \item Given all category {\bf KPIs}, {\it deep dived} into SKU level performance to come up with solutions to counter degrowth
\end{itemize}
\newgeometry{bottom = 0.3in,left = 0.4in,right = 0.4in,top = 0.3in}
%\vspace{-5pt}
\hspace*{-15pt}\textbf{Machine Learning Intern} \hfill{\sl \small (May - June '19)}\\
{\it Guide: Prof. Vipul Arora}\hfill{\sl \small \textsc{Indian Institute of Technology, Kanpur}}\\
\vspace{-19pt}
\begin{itemize}[itemsep = -1 mm, leftmargin=*]
    \item \underline{\it Analysis of ML Algorithms for Spam Email Classification in Python}: Anaylzed SVMs and Neural Networks before\\implementing {\bf Naive Bayes} and {\bf KNN} on numerous data sets using {\it Keras}, Pandas, Numpy and {\it Scikit-learn}
      \item Compared accuracies for various data sets and {\bf categorised the best method} for each data set
\end{itemize}
\vspace{-5pt}
\textbf{Software Engineering Intern} \hfill{\sl \small (Nov - Dec '18)}\\
{\it Guide: Mr. Mohsin Ali (Project Manager)}\hfill{\sl \small \textsc{Citytech Software Pvt. Ltd.}}\\
\vspace{-19pt}
\begin{itemize}[itemsep = -1 mm, leftmargin=*]
  \item Configured and enhanced a {\bf chatbot} for Employee Leave Applications using {\bf Microsoft LUIS} after a {\sl comparative study} with {\it Google Dialog Flow}. Helped in introducing \textsl{Voice to Text feature} (using Bing API) from Microsoft Azure
  \item Researched on {\bf Human Resource Automation} and developments in \textit{Google Assistant, IBM Watson, Alexa and Cortana}
\end{itemize}

\vspace{-20pt}
\section*{\LARGE Projects \& Key Assignments\xfilll[0pt]{0.5pt}}
\vspace{-9pt}
\textbf{Textgraphs-14 | COLING 2020} \hfill{\sl \small \textsc{NLP Conference Workshop} $\bigm|$ (July '20 - Present)}\\
\vspace{-19pt}
\begin{itemize}[itemsep = -1 mm, leftmargin=*]
    \item Shared Task on \textsc{Multi-Hop Inference for Explanation Regeneration}%\hfill{\sl \small \textsc{Personal Project}}
    \vspace{-7pt}
    \begin{itemize}[itemsep = -0.75 mm, leftmargin=*]
     % \begin{small}
      \item Optimizing the {\it tf.idf} baseline to construct gold explanations for elementary science questions, using the WorldTree corpus
      \item Linking facts (Textgraph) based on {\it lexical overlap} to select {\sl next-hop} explanations and using \textsc{BERT} as a reranking model
      %\end{small} $\bigm|$ \textsc{}
    \end{itemize}
\end{itemize}
\vspace{-5pt}
\textbf{Scholary Document Processing | EMNLP 2020} \hfill{\sl \small \textsc{NLP Conference Workshop} $\bigm|$ (Apr '20 - Present)}\\
\vspace{-19pt}
\begin{itemize}[itemsep = -1 mm, leftmargin=*]
    \item \textsc{LongSumm: }Shared Task for \underline{\textsl{Generating long summaries of scientific documents:}} %\hfill{\sl \small \textsc{Personal Project}}
    \vspace{-7pt}
    \begin{itemize}[itemsep = -0.75 mm, leftmargin=*]
     % \begin{small}
      \item Parsed PDFs to extract individual sections of scientific papers using \textsc{Grobid} and \textit{Beautiful Soup}, in \textbf{Python}
     \item Combining \textit{abstractive} \& {\it extractive} summarization and exploring \textsl{Reinforcement Learning} techniques for summarization 
      \item Used \textsc{Rouge} between sentences in target summary and document, to generate target sectional summaries
     \item Building models based on \textbf{PreSumm} (\textbf{SciBERT} with pretrained encoders), to summarize each section seperately
      %\end{small}
    \end{itemize}
\end{itemize}
\vspace{-5pt}
\textbf{Spanning Tree Protocol} $\bigm|$ {\it Prof. Varsha Apte | Computer Networks}\hfill{\sl \small \textsc{Course Project} $\bigm|$ (Feb - Mar '20)}\\
%{\it Guide: Prof. Varsha Apte | Computer Networks} \hfill{\sl \small \textsc{IIT Bombay}}\\
\vspace{-19pt}
\begin{itemize}[itemsep = -1 mm, leftmargin=*]
   \item Simulated the network bridge topology as a \textit{distributed system} of nodes, communicating via messages, in \textbf{C++}
    \item Configured nodes to run the protocol and agree upon a \textit{loop-less} logical topology to prevent a \textit{broadcast storm}
\end{itemize}
\vspace{-5pt}
\textbf{SAT Solver} $\bigm|$ {\it Prof. Ashutosh Gupta | Logic for CS}\hfill{\sl \small \textsc{Course Project} $\bigm|$ (Jan - Feb '20)}\\
%{\it Guide: Prof. Ashutosh Gupta | Logic for CS}\hfill{\sl \small \textsc{IIT Bombay}}\\
\vspace{-19pt}
\begin{itemize}[itemsep = -1 mm, leftmargin=*]
   \item Designed a SAT Solver using \textbf{z3 in Python}, to check satisfiability in CNF (Conjunctive Normal Form)
    \item Solved the \textit{NQueens and Sudoku} problems with the designed solver, using \textbf{DPLL} (a backtracking algorithm)
\end{itemize}
\vspace{-5pt}
\textbf{Google Forms and Survey Management} \hfill{\sl \small (Sept - Nov '19)}\\
{\it Guide: Prof. Amitabha Sanyal | Software Systems}\hfill{\sl \small \textsc{Course Project}}\\
\vspace{-19pt}
\begin{itemize}[itemsep = -1 mm, leftmargin=*]
    \item Designed own Form and Survey Management system like Google Forms with \textit{user authentication}
    \item Allowed \textit{\underline{modular}} question design (paragraph, file upload, dropdown, checkbox, radio button), form validation (constraints on answers like alphanumeric, range, email-ID, .pdf only), adding \textbf{collaborators} and shareable forms (surveys \& quizzes)
    \item Data analyzable by plotting of numerics (\textbf{Matplotlib}), learning dependencies in responses and summarized presentation of subjective answers. Used \textbf{Django} for backend, \textbf{Sqlite3} for database structure, \textbf{Bootstrap} for responsiveness
\end{itemize}
\vspace{-5pt}
\textbf{Efficient Memory Allocator} $\bigm|$ {\it Prof. Ajit Diwan | Data Structures} \& {\it Algorithms}\hfill{\sl \small \textsc{Course Project} $\bigm|$ (Aug - Sept '19)}\\
%{\it Prof. Ajit Diwan} $\bigm|$ {Data Structures} \& {\it Algorithms}\hfill{\sl \small \textsc{IIT Bombay}}\\
\vspace{-19pt}
\begin{itemize}[itemsep = -1 mm, leftmargin=*]
  \item Designed a simulator in \textbf{C++} for efficient dynamic memory allocation of processes using the \textbf{first-fit strategy}
   \item Handled allocation, deallocation and termination requests for upto $10^{\text{6}}$ process requests simultaneously
\end{itemize}
\vspace{-5pt}
\textbf{PCA for Fruit Image Generation and MNIST} \hfill{\sl \small (Oct - Nov '19)}\\{\it Guide: Prof. Suyash Awate | Data Analysis}\hfill{\sl \small \textsc{Course Project}}\\
\vspace{-19pt}
\begin{itemize}[itemsep = -1 mm, leftmargin=*]
   \item Plotted closest representations of RGB fruit images, using \textit{\underline{Principal Component Analysis}}, fitting a \textit{MultiVariate Gaussian}
    \item Generated new images by random sampling (representative of the dataset), using the closest representations, in \textbf{MATLAB}
    \item Performed \textsc{Principal Component Analysis} on digit images (MNIST) to visualize principal modes of variation about the mean (by fitting a \textit {MultiVariate Gaussian}), in \textbf{MATLAB} and decided on number of \textit{degrees of freedom} of each digit
    \item Attributed reasons to less number of principal modes and concluded behavioural patterns in writing digits
\end{itemize}
%\vspace{-5pt}
%\textbf{PCA on MNIST data} \hfill{\sl \small (Oct - Nov '19)}\\{\it Guide: Prof. Suyash Awate | Data Analysis}\hfill{\sl \small \textsc{IIT Bombay}}\\
%\vspace{-19pt}
%\begin{itemize}[itemsep = -1 mm, leftmargin=*]
%   \item Performed \textsc{Principal Component Analysis} on digit images (MNIST) to visualize principal modes of variation about the mean (by fitting a \textit {MultiVariate Gaussian}), in \textbf{MATLAB} and decided on number of \textit{degrees of freedom} of each digit
%    \item Attributed reasons to less number of principal modes and concluded behavioural patterns in writing digits
%\end{itemize}
\vspace{-5pt}
\textbf{Non Parametric Estimation \& Cross Validation} \hfill{\sl \small (Sept - Oct '19)}\\{\it Guide: Prof. Ajit Rajwade | Data Analysis}\hfill{\sl \small \textsc{Course Project}}\\
\vspace{-19pt}
\begin{itemize}[itemsep = -1 mm, leftmargin=*]
  \item Compared non paramteric estimation methods (histograms \& Kernel Density Estimation), anaylzed the \textit{rate of convergence}
    \item Implemented Cross Validation in \textbf{MATLAB} (\textsl{bandwidth selection} giving maximum joint likelihood \& minimum deviation) %between the empirical \& actual PDF
\end{itemize}
%\vspace{-5pt}
%\textbf{Screen Printing | Three Electrode System} \hfill{\sl \small (Jan - Apr '19)}\\{\it Guide: Prof. Parag Bhargava | Materials and Technology}\hfill{\sl \small IIT Bombay}\\
%\vspace{-17pt}
%\begin{itemize}[itemsep = -0.75 mm, leftmargin=*]
%  \item Investigated {\underline{\it Rheological}} properties of Carbon paste and analyzed {\it Cyclic Voltametry} for three electrode system  
%  \item Extensively made a three electrode system {\bf cheaper} then any other in the market using carbon and silver paste
%\end{itemize}
%\vspace{-5pt}
%\textbf{Image Reconstruction \& Compression} \hfill{\sl \small (Aug '19)}\\
%{\it Guide: Prof. Amitabha Sanyal | Software Systems}\hfill{\sl \small \textsc{IIT Bombay}}\\
%\vspace{-19pt}
%\begin{itemize}[itemsep = -1 mm, leftmargin=*]
%  \item Transformed {\bf distorted} images by cleaning out noises such as {\it salt and pepper} noise using Numpy and Scipy
%  \item Used {\it KMeans++} algorithm to flatten out coloured images across several K values to get the {\bf Enhanced} Image  
%\end{itemize}

%%%%%%%%%%%%%%%%%%%%%%%%%%%%%%%%%%%%%%%%%%%%%%%%%%%  POSITIONS OF RESPONSIBILITY   %%%%%%%%%%%%%%%%%%%%%%%%%%%%%%%%
\vspace{-20pt}
\section*{\LARGE Positions of Responsibility\xfilll[0pt]{0.5pt}}
\vspace{-10pt}
\textbf{Teaching Assistant} $\bigm|$ {\it CS 101 - Computer Programming} \& {\it Utilization} $\bigm|$ {\it Prof. Purushottam Kulkarni} \hfill{\sl \small (July - Nov '19)}
%\vspace{1pt}
\\
%{\it CS 101 - Computer Programming} \& {\it Utilization} $\bigm|$ {\it Prof. Purushottam Kulkarni} \hfill{\sl \small \textsc{IIT Bombay}}\\
\vspace{-19pt}
\begin{itemize}[itemsep = -1 mm, leftmargin=*]
\item \textbf{Only sophomore} to be selected for TA-Ship on the basis of {\bf academic prowess} in the subject. Involved in teaching and assisting students within and outside lab hours,  with problems and conceptual doubts on a {\bf one-to-one basis}
\end{itemize}
\vspace{-5pt}
\textbf{Interact Coordinator | Community Service} $\bigm|$ {\it Rotary Club of Calcutta Visionaries}\hfill{\sl \small (2017-18)}
%\vspace{1pt}
\\
%\hfill{\sl \small \textsc{Rotary International}}\\
\vspace{-19pt}
\begin{itemize}[itemsep = -1 mm, leftmargin=*]
    \item Coordinated {\bf blood camps, health camps}, eye camps, newspaper collection drives, {\bf organised sports} for {\it village children} in the Sunderbans. Conducted free {\bf computer classes} for {\it underpriveleged} children of Sambhu Sadan Vidayala, Kolkata
\end{itemize}
\vspace{-5pt}
\textbf{Events Coordinator, Techfest} $\bigm|$ {\it Asia's largest Science and Technology Festival} \hfill{\sl \small \textsc{IIT Bombay} $\bigm|$ (2019-20)}
%\vspace{1pt}\\
\\
%\hfill{\sl \small \textsc{IIT Bombay}}\\
\vspace{-19pt}
\begin{itemize}[itemsep = -1 mm, leftmargin=*]
\item Spearheaded a team of {\bf 15+} in conceptualizing and organizing {\bf Technoholix}, featuring {\sl International} performances. Organized {\bf PAN India} workshops on investment education with {\sl NISM, NSE} \& {\sl SEBI}, under the {\it Financial Literacy Initiative}
% , part of the {\it Techfest World MUN} 2020 team
\end{itemize}

%%%%%%%%%%%%%%%%%%%%%%%%%%%%%%%%%%%%%%%%%%%%%%%%%%%%%%  TECHNICAL SKILLS   %%%%%%%%%%%%%%%%%%%%%%%%%%%%%%%%%%%%%%%%

\vspace{-20pt}
\section*{\LARGE Technical Skills\xfilll[0pt]{0.5pt}}
\vspace{-10pt}
\begin{tabular}{p{3cm} p{17cm}}
    \textbf{Languages} & \textsc{C++, Python, Java, Bash, Matlab}  \\
    \textbf{ML Libraries} & \textsc{Pytorch, Keras, TensorFlow, Ray, Hyperopt, Scikit-Learn, Rouge, Grobid}  \\
    \textbf{Web Tools} & \textsc{HTML5, CSS3, Javascript, Bootstrap, Django, Sqlite3, Markdown} \\
    \textbf{Software Tools} & \textsc{Git, \LaTeX, Sed, Awk, Makefiles, Scipy, ns3, Wireshark, Proverif}\\
\end{tabular}
%%%%%%%%%%%%%%%%%%%%%%%%%%%%%%%%%%%%%%%%%%%%%%%%%%%%%%%%  KEY COURSES UNDERTAKEN  %%%%%%%%%%%%%%%%%%%%%%%%%%%%%%%
% \vspace{-13pt}
% \section*{\LARGE Key Courses Undertaken\xfilll[0pt]{0.5pt}}
% \vspace{-9pt}
% \hspace{-8pt}
%   \begin{tabular}{p{32mm} p{15cm}}
%     %\setlength\extrarowheight{-8pt}
%     \textbf{Computer Science \& Mathematics} & Data Structures and Algorithms, Discrete Structures, Abstractions \& Paradigms*, Data Analysis and Interpretation, Software Systems, Computer Programming \& Utilization, Computer Networks*, Digital Logic Design*, Logic for CS*, Calculus, Linear Algebra, Differential Equations\\
%     %\vspace{0.6 mm}\textbf{Mathematics} & \vspace{0.6 mm}Calculus, Linear Algebra, Differential Equations\\ % Confirm the Stats course
%     \vspace{-5pt}\textbf{Others} & \vspace{-5pt}Quantum Physics, Electricity and Magnetism, Introduction to Electrical and Electronics Circuits\\
%   \end{tabular}
%   \vspace{-2.5pt}\\
%   % \hspace*{115mm}{\sl *{\it to be completed by November 2016}} \\
%   \hspace*{108mm}{\sl {\it *: ongoing, to be completed by April 2020}}
%%%%%%%%%%%%%%%%%%%%%%%%%%%%%%%%%%%%%%%%%%%%%%%%%%%%%%%%%%%  EXTRACURRICULARS   %%%%%%%%%%%%%%%%%%%%%%%%%%%%%%%%%%

% \vspace{-15pt}
% \section*{\LARGE Extracurriculars\xfilll[0pt]{0.5pt}}
% \vspace{-8pt}
% \begin{itemize}[itemsep = -0.75 mm, leftmargin=*]
%  % \item Assisting in conceptualizing and organizing Technoholix featuring concerts and performances from \\renowned international performers and DJs and Techfest World MUN, as a Techfest 2020 Events\\ Coordinator\hfill{\sl \small (2019-20)}
%  \item Completed a year long training programme in {\it Lawn Tennis} at IIT Bombay \hfill{\sl \small (2018-19)}
%  \item Participated in {\bf DanceMania} 2k19, the Inter Hostel Dance Competition at IIT Bombay  \hfill{\sl \small (2019)}
%  \item Completed {\bf Eight} levels of Abacus in the UCMAS Academy and won a {\it Merit Certificate} for Level Six\hfill{\sl \small (2010)}
%  \item Active player of Cricket and Table Tennis at the school level \hfill{\sl \small (2016-18)}
%  \item Took part in the National Science Olympiad and won a {\it gold} medal \hfill{\sl \small (2016)}
%  \item Participated in the Spanish {\it Guitar playing competition}, Sur-Tal  \hfill{\sl \small (2012)}
%  % \item Bagged {\bf 2\textsuperscript{nd} position} in the {\bf Documentary Film Making} Inter-School Competition, held at MSMSV \hfill{\sl \small (2012)}
%  % \item Successfully completed one year as a Squash player under {\bf National Sports Organization} \hfill{\sl \small (2015-16)}

%  % \item Attended the 5-Day National Science(Vijyoshi) Camp, organised by KVPY, at IISER, Kolkata and also the 5-Day Nurturance Programme for NTSE scholars organised at RIE, Ajmer\hfill{\sl \small (2014 \& 2012)}
%  % \item Secured {\bf 2\textsuperscript{nd} position} in the Pushpa Jaipuria {\bf Mathematics Olympiad} \hfill{\sl \small (2013)}
% \end{itemize}

%Confirm Years of all

%French

\end{document}\grid
\grid
